%%
%% This is file `sample-acmlarge.tex',
%% generated with the docstrip utility.
%%
%% The original source files were:
%%
%% samples.dtx  (with options: `all,journal,bibtex,acmlarge')
%% 
%% IMPORTANT NOTICE:
%% 
%% For the copyright see the source file.
%% 
%% Any modified versions of this file must be renamed
%% with new filenames distinct from sample-acmlarge.tex.
%% 
%% For distribution of the original source see the terms
%% for copying and modification in the file samples.dtx.
%% 
%% This generated file may be distributed as long as the
%% original source files, as listed above, are part of the
%% same distribution. (The sources need not necessarily be
%% in the same archive or directory.)
%%
%%
%% Commands for TeXCount
%TC:macro \cite [option:text,text]
%TC:macro \citep [option:text,text]
%TC:macro \citet [option:text,text]
%TC:envir table 0 1
%TC:envir table* 0 1
%TC:envir tabular [ignore] word
%TC:envir displaymath 0 word
%TC:envir math 0 word
%TC:envir comment 0 0
%%
%% The first command in your LaTeX source must be the \documentclass
%% command.
%%
%% For submission and review of your manuscript please change the
%% command to \documentclass[manuscript, screen, review]{acmart}.
%%
%% When submitting camera ready or to TAPS, please change the command
%% to \documentclass[sigconf]{acmart} or whichever template is required
%% for your publication.
%%
%%
\documentclass[acmlarge]{acmart}
%%
%% \BibTeX command to typeset BibTeX logo in the docs
\AtBeginDocument{%
  \providecommand\BibTeX{{%
    Bib\TeX}}}

%% Rights management information.  This information is sent to you
%% when you complete the rights form.  These commands have SAMPLE
%% values in them; it is your responsibility as an author to replace
%% the commands and values with those provided to you when you
%% complete the rights form.
\setcopyright{acmlicensed}
\copyrightyear{2018}
\acmYear{2018}
\acmDOI{XXXXXXX.XXXXXXX}

%%
%% These commands are for a JOURNAL article.
\acmJournal{POMACS}
\acmVolume{37}
\acmNumber{4}
\acmArticle{111}
\acmMonth{8}

%%
%% Submission ID.
%% Use this when submitting an article to a sponsored event. You'll
%% receive a unique submission ID from the organizers
%% of the event, and this ID should be used as the parameter to this command.
%%\acmSubmissionID{123-A56-BU3}

%%
%% For managing citations, it is recommended to use bibliography
%% files in BibTeX format.
%%
%% You can then either use BibTeX with the ACM-Reference-Format style,
%% or BibLaTeX with the acmnumeric or acmauthoryear sytles, that include
%% support for advanced citation of software artefact from the
%% biblatex-software package, also separately available on CTAN.
%%
%% Look at the sample-*-biblatex.tex files for templates showcasing
%% the biblatex styles.
%%

%%
%% The majority of ACM publications use numbered citations and
%% references.  The command \citestyle{authoryear} switches to the
%% "author year" style.
%%
%% If you are preparing content for an event
%% sponsored by ACM SIGGRAPH, you must use the "author year" style of
%% citations and references.
%% Uncommenting
%% the next command will enable that style.
%%\citestyle{acmauthoryear}


%%
%% end of the preamble, start of the body of the document source.
\begin{document}

%%
%% The "title" command has an optional parameter,
%% allowing the author to define a "short title" to be used in page headers.
\title{Experimentation Platforms: Literature Review }

%%
%% The "author" command and its associated commands are used to define
%% the authors and their affiliations.
%% Of note is the shared affiliation of the first two authors, and the
%% "authornote" and "authornotemark" commands
%% used to denote shared contribution to the research.
\author{Innocent Ndubuisi-Obi Jr}
\email{innoobi@cs.washington.edu}
\affiliation{%
  \institution{University of Washington}
  \city{Seattle}
  \state{Washington}
  \country{USA}
}

\author{Siya Kulkarni}
\email{siyak2@uw.edu}
\affiliation{%
  \institution{University of Washington}
  \city{Seattle}
  \state{Washington}
  \country{USA}
}


%%
%% By default, the full list of authors will be used in the page
%% headers. Often, this list is too long, and will overlap
%% other information printed in the page headers. This command allows
%% the author to define a more concise list
%% of authors' names for this purpose.
\renewcommand{\shortauthors}{Trovato et al.}

%%
%% The abstract is a short summary of the work to be presented in the
%% article.
\begin{abstract}
  A clear and well-documented \LaTeX\ document is presented as an
  article formatted for publication by ACM in a conference proceedings
  or journal publication. Based on the ``acmart'' document class, this
  article presents and explains many of the common variations, as well
  as many of the formatting elements an author may use in the
  preparation of the documentation of their work.
\end{abstract}

%%
%% The code below is generated by the tool at http://dl.acm.org/ccs.cfm.
%% Please copy and paste the code instead of the example below.
%%
\begin{CCSXML}
<ccs2012>
 <concept>
  <concept_id>00000000.0000000.0000000</concept_id>
  <concept_desc>Do Not Use This Code, Generate the Correct Terms for Your Paper</concept_desc>
  <concept_significance>500</concept_significance>
 </concept>
 <concept>
  <concept_id>00000000.00000000.00000000</concept_id>
  <concept_desc>Do Not Use This Code, Generate the Correct Terms for Your Paper</concept_desc>
  <concept_significance>300</concept_significance>
 </concept>
 <concept>
  <concept_id>00000000.00000000.00000000</concept_id>
  <concept_desc>Do Not Use This Code, Generate the Correct Terms for Your Paper</concept_desc>
  <concept_significance>100</concept_significance>
 </concept>
 <concept>
  <concept_id>00000000.00000000.00000000</concept_id>
  <concept_desc>Do Not Use This Code, Generate the Correct Terms for Your Paper</concept_desc>
  <concept_significance>100</concept_significance>
 </concept>
</ccs2012>
\end{CCSXML}

\ccsdesc[500]{Do Not Use This Code~Generate the Correct Terms for Your Paper}
\ccsdesc[300]{Do Not Use This Code~Generate the Correct Terms for Your Paper}
\ccsdesc{Do Not Use This Code~Generate the Correct Terms for Your Paper}
\ccsdesc[100]{Do Not Use This Code~Generate the Correct Terms for Your Paper}

%%
%% Keywords. The author(s) should pick words that accurately describe
%% the work being presented. Separate the keywords with commas.
\keywords{Do, Not, Us, This, Code, Put, the, Correct, Terms, for,
  Your, Paper}

\received{20 February 2007}
\received[revised]{12 March 2009}
\received[accepted]{5 June 2009}

%%
%% This command processes the author and affiliation and title
%% information and builds the first part of the formatted document.
\maketitle

\section{Linkedin}
T-REX Experimentation Platform
\begin{itemize}
    \item T-REX - LinkedIn's Targeting, Ramping, and Experimentation platform.
    \item Supports up to 41,000 A/B tests at the same time for 700M+ members.
    \item Started small but evolved over a decade due to:
    \begin{itemize}
        \item Company growth
        \item More available data
        \item Shift towards experimentation in development
    \end{itemize}
\end{itemize}
Challenges
\begin{itemize}
    \item Users were divided into 1,000 buckets based on their ID
    \item Issues:
    \begin{itemize}
        \item Poor randomization = unreliable results.
        \item No central database for test management → manual bucket allocation via emails.
        \item Test definitions scattered in code → hard to debug.
        \item Deployment took hours (had to modify and push configs).
        \item Reports were manually created in spreadsheets or R - time-consuming \& error-prone.
    \end{itemize}
\end{itemize}
New System 
\begin{itemize}
    \item Needed a scalable solution for:
    \begin{itemize}
        \item More employees using A/B testing.
        \item More tests running at the same time.
        \item Growing user base.
    \end{itemize}
    \item Key Redesign Concepts:
    \begin{itemize}
        \item T-REX "Test" = Represents a hypothesis or feature.
        \item T-REX "Experiment" = A stage in testing (e.g., rollout at 5\%, then 25\%, 50\%, etc.).
    \end{itemize}
    \item Centralized UI for test management \& debugging.
    \item Deployment reduced to under 5 minutes (faster rollouts).
    \item Targeting users was built-in (no need to modify code for each new test).
    \item Lix DSL = LinkedIn's domain-specific language for defining experiments.
    \item Caching System to improve performance:
    \begin{itemize}
        \item Lix client tries to evaluate tests from memory (~98-99\% success).
        \item If missing, fetches from backend (~93\% success).
        \item Final fallback = LinkedIn's key-value store Venice.
    \end{itemize}
    \item Less than 0.2\% of requests reach storage → fast processing.
    \item Experimentation engine sped up 20x.
    \item Better statistical methods and UI for analysis.
    \item Further improvements in offline A/B data pipelines.
\end{itemize}

\section{Netflix}
\begin{itemize}
  \item Redesigned to be Science Centric
  \item Scalable, performant, cost efficient, trustworthy, usable, extensible
  \item Python, R, C++ as standard languages of the platform, because v familiar to data scientists and has massive collection of libraries
  \item Can run prod code in local environment
  \item New statistical methods have been added to platform since it was reimagined (quantile bootstrapping, regression)
  \item Engineers have more time to focus on platform itself, and work on improving it, good for scalability
  \item Want to be generic and extensible
  \item Twitter supports 3 types of metrics - built in metrics tracked for all experiments, event-based metrics, metrics that are owned and generated by the engineers
  \item Microsoft ExP Platform: trustworthiness and scalability emphasized
  \begin{itemize}
    \item 4 main components: experimentation portal, experiment execution service, log processing service, analysis service
    \item Can also deep dive post experiment for understanding changes in metrics
  \end{itemize}
  \item Research method: collected data from important meetings, grouped possible design changes, staged the changes in a way that would make the foundations strong
  \begin{itemize}
    \item Categories: requirements, architecture changes, software libraries, performance improvements, statistical methods, causal inference modeling
  \end{itemize}
  \item Software architecture: 
  \begin{itemize}
    \item 3 main phases
    \begin{itemize}
      \item Data collection: metrics repo in python, generated sql, take data from AB test logs
      \item Statistical Analysis: causal models
      \item Visualization: XP Viz
      \item From here, XP Platform API delivers these visuals to the frontend, Ablaze. Also can got o Jupyter Notebook
    \end{itemize}
    \item Metrics Repo: centralization of metrics in unified way - better than everyone having their own pipelines
    \begin{itemize}
      \item Instead of the usual ETL pipeline, metric computation goes into dynamically generated SQL that runs on Spark - very fast execution, easy to change data
    \end{itemize}
    \item Causal Models: python library that has implementations of causal effects models. Easy to integrate with other stats libraries in R and Python
    \item XP Viz - library that makes it very quicksand easy to translate from stats output to visualization - can reuse visualizations in other contexts. 
    \begin{itemize}
      \item Has really good support for Jupyter Notebooks
    \end{itemize}
    \item XP Platform API: a REST API which stores results
    \item OpenFaaS: serverless computing platform that Netflix runs its jobs on. Scalable for larger loads
  \end{itemize}
  \item Impact: 
  \begin{itemize}
    \item Many more metrics contributions
    \item Since dev of causal models, Netflix can take advantage of OLS, which increased statistical power by a lot, as well as quantile bootstrapping, segment discovery, quantile regression, time series models.
  \end{itemize}
\end{itemize}




\newpage

%%
%% The next two lines define the bibliography style to be used, and
%% the bibliography file.
\bibliographystyle{ACM-Reference-Format}
\bibliography{sample-base}


%%
%% If your work has an appendix, this is the place to put it.
\appendix

\section{Research Methods}
Testing the fork
\subsection{Part One}

Lorem ipsum dolor sit amet, consectetur adipiscing elit. Morbi
malesuada, quam in pulvinar varius, metus nunc fermentum urna, id
sollicitudin purus odio sit amet enim. Aliquam ullamcorper eu ipsum
vel mollis. Curabitur quis dictum nisl. Phasellus vel semper risus, et
lacinia dolor. Integer ultricies commodo sem nec semper.

\subsection{Part Two}

Etiam commodo feugiat nisl pulvinar pellentesque. Etiam auctor sodales
ligula, non varius nibh pulvinar semper. Suspendisse nec lectus non
ipsum convallis congue hendrerit vitae sapien. Donec at laoreet
eros. Vivamus non purus placerat, scelerisque diam eu, cursus
ante. Etiam aliquam tortor auctor efficitur mattis.

\section{Online Resources}

Nam id fermentum dui. Suspendisse sagittis tortor a nulla mollis, in
pulvinar ex pretium. Sed interdum orci quis metus euismod, et sagittis
enim maximus. Vestibulum gravida massa ut felis suscipit
congue. Quisque mattis elit a risus ultrices commodo venenatis eget
dui. Etiam sagittis eleifend elementum.

Nam interdum magna at lectus dignissim, ac dignissim lorem
rhoncus. Maecenas eu arcu ac neque placerat aliquam. Nunc pulvinar
massa et mattis lacinia.

\end{document}
\endinput
%%
%% End of file `sample-acmlarge.tex'.
